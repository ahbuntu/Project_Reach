\section{Conclusions and Future Works}

The initial goal of REACH was to detect a specific grip type and perform an action as result. Although we were not able to implement it in an end-to-end manner, we believe that we have demonstrated the feasibility of such a system. It is impressive to see that the classification tree machine learning algorithm was able to accurately distinguish between 3 types of grip patterns.

\par
For future work, the following key areas were identified - 
\begin{itemize}
  \item In future prototypes, we intend to work on miniaturizing this setup and giving it the ability to wirelessly connect to phone applications (see Figure \ref{fig:hardware_mini}).
  
  \item Interface with the hardware wirelessly to collect the force sensor data. This data would then dynamically be windowed, the features extracted and the model would make a prediction, all in real-time.
  
  \item Collect a larger training dataset with the grip gestures being performed by multiple users. Explore other features that could lead to better predictions.
  
  \item Train a model that can accurately detect gestures for a large population size. This \textit{average} model could be supplemented with specific user data to increase prediction accuracy. 
  
  \item Create a user interaction scheme that dynamically responds to the grip gestures and evaluate its effectiveness.

\end{itemize}

\begin{figure}[h]
\includegraphics[width=.45\textwidth]{hardware_mini.png}
\caption{Hardware miniaturized proof of concept}
\label{fig:hardware_mini}
\end{figure}
