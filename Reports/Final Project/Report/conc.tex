\section{Conclusions and Future Works}

The initial goal of REACH was to detect a specific grip type and perform an action as result. Although we were not able to implement it in an end-to-end manner, we believe that we have demonstrated the feasibility of such a system. It is impressive to see that the classification tree machine learning algorithm was able to accurately distinguish between 3 types of grip patterns.

\par
For future work, the following key areas were identified - 
\begin{itemize}
  \item From a hardware perspective, we would like to be able to shrink the entire setup into a case for the mobile device.
  
  \item Interface with the case over a Bluetooth connection to collect the force sensor data. This data would then dynamically be windowed, the features extracted and the model would make a prediction, all in real-time on the phone.
  
  \item Collect a larger training dataset with the grip gestures being performed by multiple users. Explore other features that could lead to better predictions.
  
  \item Create a user interaction scheme that dynamically responds to the grip gestures and evaluate its effectiveness.

\end{itemize}