% This is "sig-alternate.tex" V2.0 May 2012
% This file should be compiled with V2.5 of "sig-alternate.cls" May 2012
%
% This example file demonstrates the use of the 'sig-alternate.cls'
% V2.5 LaTeX2e document class file. It is for those submitting
% articles to ACM Conference Proceedings WHO DO NOT WISH TO
% STRICTLY ADHERE TO THE SIGS (PUBS-BOARD-ENDORSED) STYLE.
% The 'sig-alternate.cls' file will produce a similar-looking,
% albeit, 'tighter' paper resulting in, invariably, fewer pages.
%
% ----------------------------------------------------------------------------------------------------------------
% This .tex file (and associated .cls V2.5) produces:
%       1) The Permission Statement
%       2) The Conference (location) Info information
%       3) The Copyright Line with ACM data
%       4) NO page numbers
%
% as against the acm_proc_article-sp.cls file which
% DOES NOT produce 1) thru' 3) above.
%
% Using 'sig-alternate.cls' you have control, however, from within
% the source .tex file, over both the CopyrightYear
% (defaulted to 200X) and the ACM Copyright Data
% (defaulted to X-XXXXX-XX-X/XX/XX).
% e.g.
% \CopyrightYear{2007} will cause 2007 to appear in the copyright line.
% \crdata{0-12345-67-8/90/12} will cause 0-12345-67-8/90/12 to appear in the copyright line.
%
% ---------------------------------------------------------------------------------------------------------------
% This .tex source is an example which *does* use
% the .bib file (from which the .bbl file % is produced).
% REMEMBER HOWEVER: After having produced the .bbl file,
% and prior to final submission, you *NEED* to 'insert'
% your .bbl file into your source .tex file so as to provide
% ONE 'self-contained' source file.
%
% ================= IF YOU HAVE QUESTIONS =======================
% Questions regarding the SIGS styles, SIGS policies and
% procedures, Conferences etc. should be sent to
% Adrienne Griscti (griscti@acm.org)
%
% Technical questions _only_ to
% Gerald Murray (murray@hq.acm.org)
% ===============================================================
%
% For tracking purposes - this is V2.0 - May 2012

\documentclass{sig-alternate}
\usepackage{graphicx}
\usepackage{caption}
\usepackage{subcaption}
\usepackage{multirow}
\usepackage{enumitem}
\usepackage[]{algorithm2e}


\begin{document}
%
% --- Author Metadata here ---
\conferenceinfo{CSC2525}{'14 UofT, CA}
%\CopyrightYear{2007} % Allows default copyright year (20XX) to be over-ridden - IF NEED BE.
%\crdata{0-12345-67-8/90/01}  % Allows default copyright data (0-89791-88-6/97/05) to be over-ridden - IF NEED BE.
% --- End of Author Metadata ---

\title{REACH: Enabling Single-Handed Operation on Large Screen Mobile Devices}

%
% You need the command \numberofauthors to handle the 'placement
% and alignment' of the authors beneath the title.
%
% For aesthetic reasons, we recommend 'three authors at a time'
% i.e. three 'name/affiliation blocks' be placed beneath the title.
%
% NOTE: You are NOT restricted in how many 'rows' of
% "name/affiliations" may appear. We just ask that you restrict
% the number of 'columns' to three.
%
% Because of the available 'opening page real-estate'
% we ask you to refrain from putting more than six authors
% (two rows with three columns) beneath the article title.
% More than six makes the first-page appear very cluttered indeed.
%
% Use the \alignauthor commands to handle the names
% and affiliations for an 'aesthetic maximum' of six authors.
% Add names, affiliations, addresses for
% the seventh etc. author(s) as the argument for the
% \additionalauthors command.
% These 'additional authors' will be output/set for you
% without further effort on your part as the last section in
% the body of your article BEFORE References or any Appendices.

\numberofauthors{3} %  in this sample file, there are a *total*
% of EIGHT authors. SIX appear on the 'first-page' (for formatting
% reasons) and the remaining two appear in the \additionalauthors section.
%
\author{
% You can go ahead and credit any number of authors here,
% e.g. one 'row of three' or two rows (consisting of one row of three
% and a second row of one, two or three).
%
% The command \alignauthor (no curly braces needed) should
% precede each author name, affiliation/snail-mail address and
% e-mail address. Additionally, tag each line of
% affiliation/address with \affaddr, and tag the
% e-mail address with \email.
%
% 1st. author
% 1st. author
\alignauthor
Varun Perumal\\
       \affaddr{Dept. of Computer Science}\\
       \affaddr{University of Toronto}\\
       \email{varun@cs.toronto.edu}
% 2nd author
\alignauthor
Ahmadul Hassan\\
       \affaddr{Dept. of Computer Science}\\
       \affaddr{University of Toronto}\\
       \email{ahmadul.hassan@gmail.com}
% 3rd author
\alignauthor
Zahid Abul-Basher\\
       \affaddr{Mech. \& Industrial Eng.}\\
       \affaddr{University of Toronto}\\
       \email{zahid@cs.toronto.edu}
}
% There's nothing stopping you putting the seventh, eighth, etc.
% author on the opening page (as the 'third row') but we ask,
% for aesthetic reasons that you place these 'additional authors'
% in the \additional authors block, viz.

% Just remember to make sure that the TOTAL number of authors
% is the number that will appear on the first page PLUS the
% number that will appear in the \additionalauthors section.

\maketitle
\begin{abstract}

\end{abstract}

% A category with the (minimum) three required fields
\category{H.5.2}{Information interfaces and presentation}{User Interfaces---graphical user interfaces}
%A category including the fourth, optional field follows...
%\category{D.2.8}{Software Engineering}{Metrics}[complexity measures, performance measures]

\terms{Design, Experimentation, Human Factors}

\keywords{Data analytics}

\section{Introduction}
There is an unprecedented rise in popularity of large screen mobile phones with scree sizes greater than 5 inches. The benefits of larger screens and a larger battery life being the primary drivers of user adoption, however, these larger devices are difficult if not impossible to use with one hand and pose usability issues for demographics with smaller hands (especially women). The existing solutions to this include on screen functions that the user can activate to bring the screen content closer to the user's thumb. These methods however, introduce extra steps in the user's interaction with the device and can be cumbersome. We propose "Project Reach".
\par
By placing force sensors all around the rim of the phone, we can sense how the user is holding the phone and when they are straining their thumb to reach a corner. Using this information we can shift the UI closer to the operating finger. The force sensors can also be used to interact with the phone in other scenarios, for example swiping on the sides of the phone could scroll pages, or increase/decrees volume etc. With this project we intend to build the hardware, formulate UI design changes, and do basic user testing to validate our ideas.
\section{Related Work}
Many researchers have suggested that the devices should be intelligent enough to detect user's situation for better support as in \cite{Johnson:1998:UMIM} and \cite{Schmidt:1999:AIC}. For instance, \emph{ability based design} aims to find the best match between the ability of the users and the interfaces \cite{Wobbrock:2011:adcpe}. There are also researches to recognize the activity of users on devices (also known as \emph{activity recognition}). Choudhuri \emph{et al.} \cite{Choudhury:2008:MSP} built a wearable device with sensors to detect the activity of the users. In \cite{Van:2000:WhatShallWe}, Laerhoven used an accelerometer in a phone to recognize different motions of walking, climbing stairs, \emph{etc}. Schmidt \emph{et al.} \cite{Schmidt:1999:advanced} also used accelerometer but to detect both the user movement and the place of the device it self whether it is in the hand or on a table or in a suitcase. REACH falls in this research area by detecting the activity of the user from the pressure that they apply on mobile devices.




%\end{document}  % This is where a 'short' article might terminate


%
% The following two commands are all you need in the
% initial runs of your .tex file to
% produce the bibliography for the citations in your paper.
\bibliographystyle{abbrv}
\bibliography{sigproc}  % sigproc.bib is the name of the Bibliography in this case
% You must have a proper ".bib" file
%  and remember to run:
% latex bibtex latex latex
% to resolve all references
%
% ACM needs 'a single self-contained file'!
%
%APPENDICES are optional
%\balancecolumns

\end{document}
