\section{Design of REACH}

\subsection{Hardware design of REACH}


\subsection{Building the Grip Classifier}
In addition to the hardware implementation by selecting appropriate force sensors and locating them in the appropriate places around the mobile device, we also build the classifier for the hand grip. In this section, we discuss how we collect force sensor values from mobile device for training and how we implement the classifier for grip pattern detection. We then used the model to predict the pattern in realtime manner. 
\par
We selected three grip pattern classes in our prototype: Hold, Squeeze, and Reach. In Hold, the subject holds the device without any activity on device. In Squeeze, the subject is applying the squeeze-force on the device and in Reach, the subject is moving his thumb finger to reach the top of the device while he is holding the device. After we identified the grip pattern classes, we collected the training data from 3 subjects. Each subject was asked to perform Hold, Squeeze, and Reach for 3 times. The 12 force sensors around the device continuously generated data at ***Hz and we calculated three metrics; mean, variance, and delta variance as features to train a classifier model. Therefore, the training data from each grip consists from **** numerical values and the final data consists from **** numerical values for each grip pattern.
\par
For training the model,  we used the Weka (Witten \& Frank 2005) machine learning library for Android API. We used Bayesian network, Naive Bayes and Support vector machine for training the model on the collected data. We performed off-line training on a laptop and extracted the parameters of the best model to implement the realtime version of REACH.




    