\section{Software Design of REACH}
In addition to the hardware implementation by selecting appropriate force sensors and locating them in the appropriate places around the mobile device, we also build the classifier for the hand grip. In this section, we discuss how we collect force sensor values from mobile device for training and how we implement the classifier for grip pattern detection. We then used the model to predict the pattern in realtime manner. 
\par
For training the model,  we used the Weka (Witten \& Frank 2005) machine learning library for Java. We used Bayesian network, Naive Bayes and Support vector machine machine algorithms for training the model on the collected data. We performed off-line training on a laptop and extracted the parameters of the best model to implement the realtime version of REACH.

\subsection{Tap Detection as Proxy}
While waiting for the hardware to be completed, we decided to perform some experiments to provide insight into how to  classify the grip patterns. Tap detection was chosen as a proxy for grip pattern detection. We used the accelerometer data available from an Android phone to try and classify instances when a user performs a single tap on the back of the device. 
\par
The accelerometer data from 3 axis (x, y, z) is analogous to the force data available from the 16 sensors for grip detection. Similarly, both accelerometer and force data will change when an action is performed by the user. The only difference is that the accelerometer data changes depending on the orientation of the phone, while the force data is unaffected. To mitigate this, all data collected and analyzed for tap detection focused on keeping the phone in portrait orientation, the y-axis facing away from the floor and the z-axis facing towards the user.

\subsubsection{Determine Window Size}
The Android sensor collects accelerometer data on a periodic basis. This sampling interval is neither guaranteed nor specified by the Android SDK, but using the default mode, it was found to be 20ms on \textit{average}. After analyzing the data, it was noticed that the characteristic pattern of a single tap lasted for a duration of 400ms on average. This duration is defined as the \textit{window duration}.
\par
Dividing the window duration by the sampling interval gives us a \textit{window size} of 20. The accelerometer data is partitioned into windows with the specifications above, and implemented as a \textit{sliding window}. This means that the starting time for each subsequent window differs by the sampling interval. Contrast this against a \textit{jumping window}, where the start time for subsequent windows would be the window duration. Using the \textit{sliding window} protocol ensures that a possible tap is not missed during the evaluation process.
\par
Picking the right window size is crucial since it has a direct impact on the accuracy of the classification model. If the \textit{window size} is too small, then the complete characteristic pattern of a tap will not be captured, leading to incorrect classifications. However, using a window size that is too large will capture extraneous noisy data that will also reduce the classification accuracy. This topic will again be explored for grip pattern detection.

\subsubsection{Determine Features}
Upon observing the data, it was noticed that acceleration values of the x, y and z axis all changed when the tap action was performed. The \textit{inter}-window change was captured by calculating the \textit{mean} for a window, and this was used as one of the features for training the model. There was a visible change in the acceleration values within the window duration, and this \textit{intra}-window change was represented by calculating the \textit{variance} of a window. 

\subsubsection{Classify a Tap}
Even though a windowing principle was used, windows clustered near a tap duration all show a similar characteristic pattern, albeit time-shifted. Therefore while performing the manual classification of the training data, we decided to label, on average, 20 windows as a single tap. This means that when a tap is being evaluated with real-time data, we would expect 20 back-to-back windows all to be predicted as a tap. Once such a scenario is detected, we would report a successful tap as being detected.

\subsubsection{Tap Model Performance}

\subsection{Grip Classifier}
We selected three grip pattern classes in our prototype: None, Squeeze, and Reach. In None, the subject holds the device without performing any activity. In Squeeze, the subject is applying a squeeze-force on the device and in Reach, the subject is moving his thumb finger to reach the top of the device while holding the device. After we identified the grip pattern classes, we collected the training data from 1 subject. The individual was asked to perform Hold, Squeeze, and Reach every 15 seconds for a duration of 1 minute. This process was repeated 3 times. 
\par
The 12 force sensors around the device continuously reported data every 1 ms on average. The logged information was then calculated  and we calculated three metrics; mean, variance, and delta variance as features to train a classifier model. Therefore, the training data from each grip consists from **** numerical values and the final data consists from **** numerical values for each grip pattern.



    